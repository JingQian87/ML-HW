\documentclass[twoside,11pt]{homework}
\usepackage{listings}
\usepackage{xcolor}
\usepackage{graphicx} 

\coursename{COMS 4771 Machine Learning (2018 Fall)} 

\studname{Jing Qian}    % YOUR NAME GOES HERE
\studmail{jq2282@columbia.edu}% YOUR UNI GOES HERE
\hwNo{2}                   % THE HOMEWORK NUMBER GOES HERE
\date{\today} % DATE GOES HERE


\begin{document}
\maketitle

%%%%%%%%%%%%%%%%%%%%%%%%%%%%%%%%%
\section*{Problem 3}
\color{blue} (Jing, Nov 14)
\color{black}
\subsection*{(i)}
%Since $b \in \{0, 1\}^p$, there are $2^p$ possible $b$s.
%Since the entries of $A$ are picked uniformly at random, the probability of $x_i$ hashes to any $b$ is equal and hence $1/2^p$.
%\color{red} what's the point of the hint?
%\color{black}
%For the convenious for writing matrix equations, we slightly modify the definition: using the notation  $x^{(1)}, \cdots, x^{(m)} \in \{0, 1\}^n$instead of  $x_1, \cdots, x_m \in \{0, 1\}^n$.
%So 
With the matrix $A$, binary vector $x_i$ is hashed to $b$ could be expressed as following:
%
\begin{equation}
\begin{split}
b &= A x_i \\
\left[\begin{matrix} b^1 \\b^2\\ \cdots \\b^p \end{matrix} \right]&= \left[\begin{matrix} A^1  \\A^2 \\ \cdots \\A^p \end{matrix} \right]  x_i\\
&= \left[\begin{matrix} (\sum\limits_{l=1}^n A^{1l} x_i^l)\ \mathrm{mod}\ 2\\ (\sum\limits_{l=1}^n A^{2l} x_i^l)\ \mathrm{mod}\ 2\\ \cdots \\ (\sum\limits_{l=1}^n A^{pl} x_i^l)\ \mathrm{mod}\ 2 \end{matrix} \right]
\end{split}
\end{equation}

Take the $t$-th element of $b$ as an example: $b^t =  ((\sum\limits_{l=1}^n A^{tl}x_i^l)\  \mathrm{mod}\  2)$.
Since $x^i \in \{0, 1\}^n$, supposing there are $n_i$ nonzero elements in $x_i$, then $\sum\limits_{l=1}^n A^{tl} x_i^l$ is actually to randomly pick nonzero elements from the $n_i$ nonzero elements in $x_i$.
And $b^t =  ((\sum\limits_{l=1}^n A^{tl} x_i^l)\ \mathrm{mod}\ 2)$ is in fact describing whether we pick even or odd number of nonzero elements from all the nonzero elements in $x_i$.
Since $A^t$ is picked uniformly at random, the probability of the $t$-th entry of $b$, $b^t =  0$ is $1/2$ and that of $b^t  = 1$ is also $1/2$.
And this is true for any $t$ between 1 and $p$. 
Because different entries of matrix $A$ are independent with each other, 
Prob($x_i \rightarrow b$) = $\prod\limits_{t=1}^p \mathrm{Prob}(A^t x_i\rightarrow b^t) = \prod\limits_{t=1}^p 1/2 = 1/2^p.$

\subsection*{(ii)}
We could do this problem in two methods.
\\\\
METHOD 1:ß
From part (i), the probability of $x_i$ hashing to any $b$ is $1/2^p$, the probability of $x_j$ hashing to any $b$ is $1/2^p$.
So the probability of $x_j$ hashing to the same vector that $x_i$ is hashing to is $1/2^p$.
\\\\
METHOD 2:
$x_i$ and $x_j$ hash to the same vector means: $A x_i = A x_j$:
%
\begin{equation}
\begin{split}
\left[\begin{matrix} (\sum\limits_{l=1}^n A^{1l} x_i^l)\ \mathrm{mod}\ 2\\ (\sum\limits_{l=1}^n A^{2l} x_i^l)\ \mathrm{mod}\ 2\\ \cdots \\ (\sum\limits_{l=1}^n A^{pl} x_i^l)\ \mathrm{mod}\ 2 \end{matrix} \right]
= \left[\begin{matrix} (\sum\limits_{l=1}^n A^{1l} x_j^l)\ \mathrm{mod}\ 2\\ (\sum\limits_{l=1}^n A^{2l} x_j^l)\ \mathrm{mod}\ 2\\ \cdots \\ (\sum\limits_{l=1}^n A^{pl} x_j^l)\ \mathrm{mod}\ 2 \end{matrix} \right]
\end{split}
\end{equation}
%
Then
%
\begin{equation}
\begin{split}
\left[\begin{matrix} (\sum\limits_{l=1}^n A^{1l} [x_i^l  - x_j^l])\ \mathrm{mod}\ 2\\ (\sum\limits_{l=1}^n A^{2l} [x_i^l  - x_j^l])\ \mathrm{mod}\ 2\\ \cdots \\ (\sum\limits_{l=1}^n A^{pl} [x_i^l  - x_j^l])\ \mathrm{mod}\ 2 \end{matrix} \right]
= \left[\begin{matrix} 0\\ 0\\ \cdots \\0\end{matrix} \right]
\end{split}
\end{equation}
%
which means $A(x_i - x_j) = 0$.

Supposing $x_i$ and $x_j$ have $m$ differernt elements.
Then any $A^t(x_i - x_j) = 0$ for  $(1 \le t \le p)$ means to pick even number of elements from the $m$ elements.
This is the probability of a random variable X with binomial distribution Binomial($m, p_x$) being even where $p_x = 1/2$ because this is a random and uniform case.
So:
\begin{equation}
\begin{split}
\mathrm{Prob}(A^t(x_i - x_j) = 0) &= \mathrm{Prob\ (X}_{m, p_x}\mathrm{\ is\ even})\\
&= \frac{1}{2} (1 + (1-2p_x)^m) \\
&= \frac{1}{2} (1 + (1-2 \times \frac{1}{2})^m) \\
&= \frac{1}{2}
\end{split}
\end{equation}
Because diffferent elements of $b$ are independent with each other,
Prob($A(x_i - x_j) = 0$) = $\prod\limits_{t=1}^p \mathrm{Prob}(A^t(x_j - x_j) = 0) = \prod\limits_{t=1}^p 1/2 = 1/2^p.$


\subsection*{(iii)}
The probability of no collisions among the $x_i$ could be represented as following:
%
\begin{equation}
\begin{split}
\mathrm{Prob\ (no\ collisions)} &= 1 - \mathrm{Prob\ (exist\ collisions)} \\
				       &\ge 1 - \sum\limits_{1\le i <j \le m} \mathrm{Prob}(x_i, x_j \mathrm{\ collide})\\
				       &= 1 -  \sum\limits_{1\le i <j \le m} 1/2^p \\
				       &= 1 - \binom{m}{2} \frac{1}{2^p} \\
				       &= 1- \frac{m(m-1)}{2}\frac{1}{2^p} \\
				       &\ge 1 - \frac{m^2}{2}\frac{1}{2^p}
\end{split}
\end{equation}
%
 If $p \ge 2 \log_2 m$,
 %
\begin{equation}
\begin{split}
\mathrm{Prob\ (no\ collisions)} &\ge  1- \frac{m^2}{2}\frac{1}{2^p} \\
					&\ge 1 - \frac{m^2}{2} \frac{1}{m^2} \\
					&= 1 - 1/2 \\
					&= 1/2
\end{split}
\end{equation}
%
 So if  $p \ge 2 \log_2 m$, there are no collisions among the $x_i$ with probability at least 1/2.
 
\end{document} 
